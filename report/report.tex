\documentclass[11pt]{amsart}
\usepackage{geometry}                % See geometry.pdf to learn the layout options. There are lots.
\geometry{letterpaper}                   % ... or a4paper or a5paper or ... 
%\geometry{landscape}                % Activate for for rotated page geometry
\usepackage[parfill]{parskip}    % Activate to begin paragraphs with an empty line rather than an indent
\usepackage{graphicx}
\usepackage{amssymb}
\usepackage{epstopdf}

\DeclareGraphicsRule{.tif}{png}{.png}{`convert #1 `dirname #1`/`basename #1 .tif`.png}
\setcounter{secnumdepth}{5}

\title{Implementing a Scene Completion Algorithm with a Dynamic Image Resizing Extension}
\author{Matthew Conlen \\ Michael Gisi  \\ EECS 442 \\ University of Michigan}
\date{}                                           % Activate to display a given date or no date

\begin{document}
\maketitle

\begin{abstract}
write the abstract here. 
\end{abstract}

\section{Introduction}

Write the introduction here.  introduce the problem you want to solve, expain why it is important to solve it; and indicate the method you used to solve it. add a concept figure showing the overall idea behind the method you are presenting. You can cite like this \cite{Hays:2007, Karger:1992, Avidan:2007, Torralba:2006, Perez:2003}, using entries from the bibliography.bib file.

\section{Background} 
\subsection{Previous Work}
Review of previous work (i.e. previous methods that have explored a similar problem)

\subsection{Our Method}
Say why your method is better than previous work; and/or summarize the key main contributions of your work; 

\section{Technical}

\subsection{Technical Summary}
Technical part: Summary of the technical solution 

\subsection{Technical Details}
Technical part: Details of the technical solution; you may want to decompose this section into several subsections; add figures to help your explanation. 
\subsubsection{Scene Completion Algorithm}
blah blah blah

\paragraph{\sc Semantic Matching} 
another subsection

\paragraph{\sc Local Context Matching}

\paragraph{\sc Graph Cut}

Following the method of Hayes and Effros \cite{Hays:2007} we allow for dynamic expansion of the border around the area of the image which is to be replaced. That is, we allow the area to expand (up to 80 pixels in any direction) but not contract. The reasoning behind this is that it will allow for a more natural integration of the match image into the original image. Contraction of the area is not allowed because this could possibly cause specific objects that the user is trying to delete to remain in the image. 


The problem then is to identify the border which will make the match look the most natural. This formulation can be reduced to a graph cut problem. We take each pixel in the context region (the 80 pixel buffer around the original selection) and treat it as a vertex on a graph. Then, we assign weights to each of the edges. The weights are based on the following formula

\begin{displaymath}
	w_{i,j} = \left\{ 
		\begin{array}{lr}
			\triangledown diff(i,j) + (k \cdot Dist(i,hole))^3 &  i,j \in context \\
			0 &  otherwise
		\end{array}
	\right.
\end{displaymath}

where $i,j$ are pixels, $\triangledown diff(i,j)$ is the magnitude of the gradient of the SSD between the images, and $k \approx .002$ is an empirical constant presented by Wilczkowiak, et. al. \cite{Gabriel:2005}. This takes into account the visual cost of having these two pixels be from different images and also adds a penalty for pixels as they are farther away from the hole area.


We use the algorithm proposed by Karger \cite{Karger:1992} to find the minimum graph cut, and hence the label for each pixel in the context area. Karger's algorithm is relatively straightfoward


\paragraph{\sc Poisson Blending}

\subsubsection{Dynamic Resizing}

\section{Experiments}

\subsection{Preliminary Results ($\approx14000$ images)}

\subsection{Further Results ($\approx100000$ images)} 

\subsection{Dynamic Resizing}

\section{Conclusion}

\subsection{Current Results}

\subsection{Future Work}

\bibliographystyle{plain}
\bibliography{bibliography}

\end{document}  