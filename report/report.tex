\documentclass[11pt]{amsart}
\usepackage{geometry}                % See geometry.pdf to learn the layout options. There are lots.
\geometry{letterpaper}                   % ... or a4paper or a5paper or ... 
%\geometry{landscape}                % Activate for for rotated page geometry
\usepackage[parfill]{parskip}    % Activate to begin paragraphs with an empty line rather than an indent
\usepackage{graphicx}
\usepackage{amssymb}
\usepackage{epstopdf}

\DeclareGraphicsRule{.tif}{png}{.png}{`convert #1 `dirname #1`/`basename #1 .tif`.png}
\setcounter{secnumdepth}{5}

\title{Implementing a Scene Completion Algorithm with a Dynamic Image Resizing Extension}
\author{Matthew Conlen \\ Michael Gisi  \\ EECS 442 \\ University of Michigan}
\date{}                                           % Activate to display a given date or no date

\begin{document}
\maketitle

\begin{abstract}
write the abstract here. 
\end{abstract}

\section{Introduction}

Write the introduction here.  introduce the problem you want to solve, expain why it is important to solve it; and indicate the method you used to solve it. add a concept figure showing the overall idea behind the method you are presenting. You can cite like this \cite{Hays:2007, Karger:1992, Avidan:2007, Torralba:2006, Perez:2003}, using entries from the bibliography.bib file.

\section{Background} 
\subsection{Previous Work}
Review of previous work (i.e. previous methods that have explored a similar problem)

\subsection{Our Method}
Say why your method is better than previous work; and/or summarize the key main contributions of your work; 

\section{Technical}

\subsection{Technical Summary}
Technical part: Summary of the technical solution 

\subsection{Technical Details}
Technical part: Details of the technical solution; you may want to decompose this section into several subsections; add figures to help your explanation. 
\subsubsection{Scene Completion Algorithm}
blah blah blah

\paragraph{\sc Semantic Matching} 
another subsection

\paragraph{\sc Local Context Matching}

\paragraph{\sc Graph Cut}

\paragraph{\sc Poisson Blending}
A method is needed to combine the target image and the source image given by the local context matching. Simply pasting the images together may result in a jarring transition with mismatching color values. Poisson blending alleviates this operating on the source image gradient; a solution is found which minimizes the changes to the gradient while obeying boundary conditions which match the pixel values in the source on the edge of the context to the target image. The resulting image features a smooth transition with matching color values while preserving the contents of the source image.


We use poisson blending as described in Perez et. al. \cite{Perez:2003} We define f to be the unknown portion of the target image in the domain (omega) to be filled in. The known portion of the image is defined as f* and the gradient of the source image is defined as v. To successfully blend the images we must minimize the squared distance between gradient of f and v while obeying the boundary conditions fdo = f*do:
function thing
blah is the gradient operator. The above equation's solution is the unique solution to the following Poisson equation:
poisson thing
blah is the Laplacian operator and divv is the divergence of the gradient v.

\subsubsection{Dynamic Resizing}

\section{Experiments}

\subsection{Preliminary Results ($\approx14000$ images)}

\subsection{Further Results ($\approx100000$ images)} 

\subsection{Dynamic Resizing}

\section{Conclusion}

\subsection{Current Results}

\subsection{Future Work}

\bibliographystyle{plain}
\bibliography{bibliography}

\end{document}  